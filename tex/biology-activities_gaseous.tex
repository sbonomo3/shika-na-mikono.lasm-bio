\section{Gaseous Exchange and Respiration}

\subsection{Identification of Carbon Dioxide in Exhaled Air}

During respiration, living cells oxidize food substance to produce carbon dioxide. Carbon dioxide is a waste product and must be removed from the body. The blood carries carbon dioxide to the lungs where is it released in exhalation.

\subsubsection*{Learning Objectives}
\begin{itemize}
\item{To identify the presence of carbon dioxide in exhaled air.}
\end{itemize}

\subsubsection*{Materials}
Clear plastic container, plastic straw, lime water*

\begin{figure}
\begin{center}
\def\svgwidth{2cm}
\input{./img/co2-limewater.pdf_tex}
\caption{Identifying the by product \ce{CO2} in respiration}
\label{fig:CO2-limewater}
\end{center}
\end{figure}

\subsubsection*{Activity Procedure}
\begin{enumerate}
\item{Put approximately 10 mL of lime water into a clear container.}
\item{Blow exhaled air through the straw and into the limewater until a change is observed.}
\end{enumerate}

\subsubsection*{Results and Conclusion}
The addition of carbon dioxide turns limewater milky (white). This shows that carbon dioxide is present in exhaled air.

\subsubsection*{Clean Up Procedure}
\begin{enumerate}
\item{Unused lime water should be stored in an airtight labeled bottle for future use.}
\item{Used lime water contains suspended solids and should be disposed outside, not down the drain.}
\end{enumerate}

\subsubsection*{Discussion Questions}
\begin{enumerate}
\item{What change did you observe in the lime water at the end of the experiment?}
\item{What caused the change in limewater?}
\end{enumerate}

\subsubsection*{Notes}
Lime water reacts with carbon dioxide to form white calcium carbonate precipitate.
$\mathrm{CO}_{2(g)} + \mathrm{Ca(OH)}_{2(aq)} \longrightarrow \mathrm{CaCO}_{3(s)} + \mathrm{H}_2\mathrm{O}_{(l)}$

\subsection{Anaerobic Respiration}

Respiration is the production of energy through the breakdown of complex organic structures. Anaerobic respiration is respiration without oxygen. The products of anaerobic respiration are alcohol and carbon dioxide.

\subsubsection*{Learning Objectives}
\begin{itemize}
\item{To identify the products of anaerobic respiration.}
\end{itemize}

\subsubsection*{Materials}
Plastic bottle with lid, plastic syringe, delivery tube*, cotton wool, test tube*, beaker*, yeast, glucose*, water, and lime water*

\subsubsection*{Preparation Procedure}
\begin{enumerate}
\item{Make a hole on one side of plastic water bottle and connect the delivery tube, making sure there is an airtight seal.}
\item{Boil some water to remove dissolved oxygen and let it cool until.}
\item{Prepare a water bath by mixing hot and cold water. The ideal temperature is the same as human body temperature - the water should feel warm but not hot.}
\end{enumerate}

\subsubsection*{Activity Procedure}
\begin{enumerate}
\item{In the plastic bottle mix 1/2 spoon of glucose, 1/4 spoon of yeast, and approximately 50 mL of cool boiled water. Swirl the container to mix thoroughly - do not shake.}
\item{Add about 2 mL of lime water in the test tube, insert the free end of the delivery tube into the test tube, making sure that it is immersed in the lime water. Cover the test tube with cotton wool.}
\item{Dip the bottle containing the mixture of yeast and sugar in a warm water bath, make sure that the opening of the delivery tube remains submerged in the lime water.}
\item{Check periodically for bubbles passing through the lime water and note any changes that occur in the limewater.}
\item{After a change in the lime water has been noted, smell the yeast solution.}
\end{enumerate}

\subsubsection*{Results and Conclusion}
The lime water will turn milky showing the presence of carbon dioxide gas.
The students should detect a slight smell of alcohol from the mixture of glucose and yeast showing that anerobic respiration produces alcohol.

\subsubsection*{Clean Up Procedure}
\begin{enumerate}
\item{Unused lime water should be stored in a well labelled reagent bottle for further use.}
\item{Collect all the used materials, cleaning and storing items that will be used later. No special waste disposal is required.}
\end{enumerate}

\subsubsection*{Discussion Questions}
\begin{enumerate}
\item{Why is it important to boil the water used to make the yeast solution prior to this experiment?}
\item{Name the gas produced during this reaction.}
\item{Why was glucose added to the solution of yeast?}
\item{Why was the solution submerged in a warm water bath?}
\item{What smell do you detect from the mixture of glucose and yeast after the experiment?}
\item{Where is the principal of anaerobic respiration applied in a Tanzanian village? How are the products used?}
\end{enumerate}

\subsubsection*{Notes}
This principle is applied in the manufacture of alcoholic beverages, both in industry and by local brewers.
If you find that the gas is taking a long time to form, you can gently squeeze the bottle containing the yeast solution. This will force any gas formed through the delivery tube and will speed up the change in lime water. If this is done, be sure not to release the bottle before removing the cap when the straw is still submerged in limewater - otherwise air pressure will force the limewater back into the yeast solution, ending the experiment.
The smell of alcohol my be faint and hard to detect; if this is the case leave the solution for 3-4 days and smell again.