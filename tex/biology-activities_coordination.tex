\section{Coordination}
Coordination is the process of different organs working together to perform a particular function. All living organisms use coordination to respond to changes in their environment. Sense organs allow organisms to perceive changes in their environment.

The following activity is described as an example of an activity the teacher may use in the classroom to help students explore their sense organs.

\subsection{Using Sense Organs to Make Observations}
There are five sense organs in the human body that we use to make observations. We use our tongue to taste, our nose to smell, our skin to feel, our ears to hear, and our eyes to see.

\subsubsection*{Learning Objective}
\begin{itemize}
\item{To make observations with sense organs.}
\end{itemize}

\subsubsection*{Materials}
A sharp stick, a colourful flower like hibiscus or bougainvillea, salt, sugar, orange or lemon leaves, soap and water, and 2 pieces of metal

\subsubsection*{Hazards and Safety}
\begin{itemize}
\item{This activity should not be conducted in a laboratory as nothing should ever enter the mouth in a school laboratory.}
\end{itemize}

\subsubsection*{Activity Procedure}
\begin{enumerate}
\item{Instruct all students to wash their hands with soap.}
\item{Provide each group with a sharp stick, flower, small amount of sugar and salt, and a lemon or orange leaf.}
\item{Instruct one partner to close their eyes and open their mouth.}
\item{Instruct the other partner to put a very small amount of sugar and then salt on their partner's tongue. Tell the student to describe the taste of each unknown substance.}
\item{Instruct the student to keep their eyes closed and smell one of the crushed lemon/orange leaves. Guide students to describe the smell of each unknown substance.}
\item{Instruct students to touch each other with a sharp stick and describe the feeling.}
\item{Instruct students to describe the colour and shape of the flower.}
\item{Instruct all students to close their eyes. Strike the metal rods together.}
\item{Guide students to describe what they have heard.}
\end{enumerate}

\subsubsection*{Results and Conclusion}
By using our sense organs, we should be able to make hypotheses about what we are tasting, smelling, hearing, feeling, and seeing.

\subsubsection*{Clean Up Procedure}
\begin{enumerate}
\item{Collect all the used materials, cleaning and storing items that will be used later. No special waste disposal is required.}
\end{enumerate}

\subsubsection*{Discussion Questions}
\begin{enumerate}
\item{Mention the five sense organs and their functions.}
\item{Is there any relationship between the sense of smell and that of taste? Give an explanation.}
\end{enumerate}
