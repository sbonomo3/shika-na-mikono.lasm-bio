\section{Movement}
Movement is the change in the position of an organism or a part of an organism. There are two types of movement: locomotion and growth curvature. Animals, protoctista, and some bacteria use locomotion to move their whole body from place to place. Plants use growth curvature to respond to stimuli such as light, gravity and important chemicals needed for growth and survival.

These activities are conducted well in small groups over a week. 

\subsection{Investigate the Effects of Phototropism}
Tropic movement is the directional movement of a plant in response to an external stimulus. Phototropism is the growth of plant shoots towards light.

\subsubsection*{Learning Objectives}
\begin{itemize}
\item{To carry out experiments to investigate the effect of phototropism on plants.}
\end{itemize}

\subsubsection*{Materials}
Maize grain, 2 pots or containers, cotton wool, and 2 boxes

\subsubsection*{Preparation Procedure}
\begin{enumerate}
\item{Make a hole in one of the boxes on the side.}
\item{Germinate maize grains by placing them into a container with wet soil.}
\end{enumerate}

\subsubsection*{Activity Procedure}
\begin{enumerate}
\item{Cover the maize seedlings with the two boxes, one with a hole and one without a hole.}
\item{Leave the covered plants for about 2-3 days.}
\item{Uncover the boxes and observe the direction of the shoot in the two separate plants.}
\end{enumerate}

\begin{figure}[h]
\begin{center}
\def\svgwidth{8cm}
\input{./img/phototropism.pdf_tex}
\caption{A diagram of a phototropism activity.}
\label{fig:phototropism}
\end{center}
\end{figure}

\subsubsection*{Results and Conclusion}
The plant covered by the box with no light source will grow upwards. The plant covered by the box with the hole will grow towards the light.

\subsubsection*{Clean Up Procedure}
\begin{enumerate}
\item{Collect all used materials, cleaning and storing items that will be used later. No special waste disposal is required.}
\end{enumerate}

\subsubsection*{Discussion Questions}
\begin{enumerate}
\item{Define phototropism.}
\item{What is the importance of light in plant growth?}
\end{enumerate}

\subsubsection*{Notes}
The investigation shows that plant stems grow towards a light source. This shows that they are positively phototrophic. This tropism enables plant leaves to receive the maximum amount of light for photosynthesis.

\subsection{Investigation of the Effects of Hydrotropism}
Plants need water to grow and survive. Because of this, plant roots tend to grow towards the source of water. This process is termed hydrotropism.

\subsubsection*{Learning Objectives}
\begin{itemize}
\item{To carry out an experiment to investigate the effect of hydrotropism in plants.}
\end{itemize}

\subsubsection*{Materials}
Maize and bean seedlings, water bottle, knife, take-away food container, dry saw dust, and water

\subsubsection*{Preparation Procedure}
\begin{enumerate}
\item{Prepare a perforated container by cutting a water bottle in half and then punch small holes into left half of the container.}
\end{enumerate}

\begin{figure}[h]
\begin{center}
\def\svgwidth{12cm}
\input{./img/hydrotropism.pdf_tex}
\caption{Testing for the effects of hydrotropism.}
\label{fig:hydro}
\end{center}
\end{figure}

\subsubsection*{Activity Procedure}
\begin{enumerate}
\item{Put saw dust in the take-away food container.}
\item{Place the perforated container in the centre of the tray with saw dust.}
\item{Fill the container in the centre with water, so one side of the saw dust tray is wet while saw dust on the other side is dry.}
\item{Sow the seedlings in the saw dust on both sides.}
\item{Leave for 4-5 days.}
\end{enumerate}

\subsubsection*{Results and Conclusion}
After a few days all the roots of the plants growing towards the left or perforated side of the water container.

\subsubsection*{Clean Up Procedure}
\begin{enumerate}
\item{Collect all the used materials, cleaning and storing items that will be used later. No special waste disposal is required.}
\end{enumerate}

\subsubsection*{Discussion Questions}
\begin{enumerate}
\item{What is hydrotropism?}
\item{What are the effects of hydrotropism to a plant?}
\end{enumerate}

\subsubsection*{Notes}
In a carefully set up experiment, you will observe that radicals did grow towards the water. Thus, water has a greater influence on root growth than gravity.

\subsection{Investigation of Effects of Geotropism}
Geotropism is the growth of shoot away from gravity and of roots towards gravity. When the plant stem grows away from gravity it is termed negative geotropism and when plant roots grow towards gravity it is termed positive geotropism. This enables the plant to anchor its roots securely in the ground, reaching water and minerals to ensure their survival and to ensure that the stem grows upright towards the light.

\subsubsection*{Learning Objectives}
\begin{itemize}
\item{To carry out experiments to investigate the effect of geotropism in plants.}
\end{itemize}

\subsubsection*{Materials}
Germinating bean seeds, moist cotton wool, petri dish, covering lids, and cellotape

\subsubsection*{Preparation Procedure}
\begin{enumerate}
\item{Prepare a petri dishes from bottle caps or from bottoms of empty plastic bottles.}
\item{Soak bean seeds/maize grains/cow peas in water.}
\end{enumerate}

\begin{figure}[h]
\begin{center}
\def\svgwidth{8.5cm}
\input{./img/geotropism.pdf_tex}
\caption{The effects of geotropism.}
\label{fig:fish}
\end{center}
\end{figure}

\subsubsection*{Activity Procedure}
\begin{enumerate}
\item{Make two layers of moist cotton wool.}
\item{Place the germinating seeds between the two layers of moist cotton wool.}
\item{Place the seeds in a petri dishes with their radicals; one facing horizontally, one pointing vertically upwards and one pointing vertically downwards.}
\item{Cover the petri dish with a lid made from a box or bottle caps.}
\item{Place the petri dish on its edge by using cellotape in the dark cardboard.}
\item{Leave it for two days and observe the changes.}
\end{enumerate}

\subsubsection*{Results and Conclusion}
Plant roots will grow towards gravity, showing a positive response to gravity. The stems will grow away from gravity, thus showing negative geotropism.

\subsubsection*{Clean Up Procedure}
\begin{enumerate}
\item{Remove all the unwanted materials from the bench.}
\end{enumerate}

\subsubsection*{Discussion Questions}
\begin{enumerate}
\item{Define geotropism.}
\item{Why is it important to moisten the cotton wool?}
\item{Which force causes the response shown by the seedlings?}
\item{How is this response important to plant life?}
\end{enumerate}

\subsubsection*{Notes}
Plant stems grow away from gravity a process termed negative geotropism while roots towards gravity is a process termed positive geotropism.  Positive geotropism enables plants to anchor its roots in the ground, reaching water and minerals necessary for their survival.
