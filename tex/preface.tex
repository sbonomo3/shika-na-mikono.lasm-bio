\chapter*{Preface}

As the era of Alternative to Practical comes to a end, it is my hope that science teachers nationally embrace the new paradigm, that the science lesson should be student-centred, competency-based, activity-oriented, and connect with student's life experience. Every student in Tanzania should perform practical exercises, not just the few tested on national exams, but the wide range of hands-on activities teachers should employ to build a deep understanding in their students.

Educational research has identified two obstacles to the universal implementation of hands-on science education. First, many teachers themselves learned in Alternative to Practical schools and therefore lack essential experience with hands-on science. Every effort is already under way to overcome this deficiency. A national in-service training program reaches tens of thousand of educators annually, and a Teachers' Guide has already been written to explain for teachers the standard execution of dozens of hands-on activities in Biology.

The remaining challenge is a fallacy rooted in ignorance and complacency: the idea that the materials required for hands-on science teaching are unavailable to most schools. We reject the notion that science education requires expensive, imported materials. Everything required to teach modern science is already available in our villages and towns. The challenge is simply to begin.

Science belongs to Tanzania as much as any country in the world. The law of gravity respects no national boundaries; we all feel its effect and can measure its strength. Those who decry the use of locally available materials as ``stone age science" misunderstand the meaning 
of Science - that it applies universally, in any situation, with any materials. Dependence on expensive imported materials teaches students that Science is a foreign concept, to be memorized rather than understood, and that Science lacks application to daily life. Science is the birthright of humanity, as much as Language or Mathematics or Music, and the time has come to embrace what we already own.

This Companion Guide was written to equip teachers with the knowledge and skills to deliver hands-on science lessons in any school, especially those without standard science laboratories. I hope that this Companion Guide will also inspire school inspectors, examiners, curriculum developers and college tutors to increase their emphasis on the importance of hands-on education, and to reject material deficiencies as an excuse for any absence of practical work. In the same spirit, this Companion Guide seeks to expand the range of approaches to learning Biology and it is my hope that the many stakeholders in science education will embrace alternative methods that enable quality science education for every student.\\[14pt]
Prof. Hamisi O. Dihenga\\[24pt]
Permanent Secretary\\
Ministry of Education and Vocational Training\\
April 2011
