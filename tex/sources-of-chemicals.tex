\chapter{Sources of Chemicals}

The following is a list of chemicals you will need in the biology laboratory. For each we note local sources of these chemicals, low cost industrial sources of
these chemicals, methods to manufacture these chemicals at your school, and/or functional alternatives to these chemicals. We also list information like other names, common uses, and hazards. 
Chemicals are generally listed alphabetically by IUPAC name.

\begin{flushleft}
\textbf{Citric Acid}\\
IUPAC Name: 2-hydroxypropane-1, 2, 3-tricarboxlyic acid\\
Formula: \ce{C6H8O7} = \ce{CH2(COOH)COH(CHOOH)CH2COOH}\\
Local Name: Ndimu ya unga\\
Description: White crystals soluble in water\\
Use: All purpose weak acid, manufacture of Benedict's solution\\
Hazard: Keep out of eyes\\
Source: Markets, Supermarkets\\
\end{flushleft}


\begin{flushleft}
\textbf{Copper Sulfate}\\
IUPAC Name: Copper (II) Sulphate pentahydrate\\
Formula: \ce{CuSO4}\\
Local Name: Mruturutu\\
Description: white (anhydrous) or blue (pentahydrate) crystals\\
Use: Manufacture of Benedict's solution, test for Proteins\\
Source: Local medicine supply shops\\

\begin{flushleft}
\textbf{Gentian Violet (GV)}
\end{flushleft}
\vspace{-10pt}
Description: Purple Liquid\\
Uses: Staining xylem cells\\
Sources: Pharmacies or hospitals\\

\begin{flushleft}
\textbf{Glucose}
\end{flushleft}
\vspace{-10pt}
Formula: \ce{C6H12O6}\\
Description: White powder\\
Uses: Food test\\
Sources: Shops or pharmacies\\
Note: For food tests, the vitamins added to most glucose products will not cause
a problem.

\begin{flushleft}
\textbf{Iodine}
\end{flushleft}
\vspace{-10pt}
Formula: \ce{I2}$_{(s)}$\\
Description: Brown liquid\\
Uses: Food test for starch and lipids\\
Sources: Pharmacies
Note: Pyrodine iodine tincture without ethanol is the best option. An iodine tincture containing ethanol might not work for some uses.

\begin{flushleft}
\textbf{Sodium Carbonate}
\end{flushleft}
\vspace{-10pt}
Formula: \ce{Na2CO3}\\
Local name: Soda ash, washing soda\\
Description: White powder completely soluble in water\\
Use: Manufacturing Benedict's solution\\
Hazard: Caustic, corrosive\\
Source: Commercial and industrial chemical supply companies or Batik manufacturers\\

\begin{flushleft}
\textbf{Sodium Hydroxide}
\end{flushleft}
\vspace{-10pt}
Formula: \ce{NaOH}\\
Local name: Caustic soda\\
Description: White deliquescent crystals \\
Uses: Food tests for protein, absorbs carbon dioxide in photosynthesis experiments\\
Hazard: Corrodes metal, burns skin, and can blind if it gets in to the eyes\\
Source: Industrial supply shops, supermarkets, hardware stores (drain cleaner)\\

\begin{flushleft}
\textbf{Sodium Hydrogen Carbonate}
\end{flushleft}
\vspace{-10pt}
Formula: \ce{NaHCO3}\\
Local name: Baking soda\\
Description: White powder \\
Uses: To add \ce{CO2} in photosynthesis experiments\\
Hazard: Corrodes metal, burns skin, and can blind if it gets in to the eyes\\
Source: Industrial supply shops, supermarkets, hardware stores (drain cleaner)
