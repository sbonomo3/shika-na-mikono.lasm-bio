\chapter*{Background}

\subsection*{Motivation for Writing this Companion Guide}

Quality science education requires students to perform experiments with their own hands. Unfortunately, research on the situation of secondary science education shows that many students do not perform such experiments. This is due to several factors, all of which can be addressed.

First, many teachers themselves do not have enough experience with science experiments, largely due to the absence of practical education when they themselves were students. Teacher training programs and the new Biology Teachers' Guide seek to address this shortcoming.

Second, this lack of experience leads to not enough [find a better word] confidence in trying new experiments. Science can only be learned through experimentation -- just as students must perform activities to truly understand the material on the syllabus, so too must teachers. The teacher using this book is strongly encouraged to perform every one of these experiments to deepen his or her fundamental understanding of Biology.

Third, most schools lack traditional laboratory facilities. Many educators therefore assume that this means hands-on activities are impossible.

To address this misconception, the Ministry of Education and Vocational Training has decided to prepare this Biology Teacher's Practical Guide. The objective is to ensure that all secondary school Biology teachers can conduct practical work even if they do not have access to a standard Biology laboratory. Specifically, this book demonstrates that many quality hands-on science experiments are possible with very basic materials. The experiments in these pages require materials available in villages or, at worst, in a regional capital. Standard laboratory materials certainly add value to science teaching; this book merely makes it clear that they are not required for quality education.

\subsection*{Procedures Followed in Developing the Companion Guide}

This Companion Guide builds on the work of the Biology Teachers' Guide. In preparation for the Teachers' Guide, educators and subject experts identified activities for most of the topics on the ordinary level syllabus. To prepare this Companion Guide [the difference between these two is still not clear], a team of secondary school teachers and science experts from TIE devised methods for performing the activities of the Biology Teachers' Guide using low cost and locally available materials.

\subsection*{Description of the Companion Guide}

Practical investigations address specific syllabus content. Each topic begins with a short summary of relevant syllabus material. Each activity is introduced in that context as a method for students to experiment with the topic of the day. Each activity then states clearly its objectives. Generally, these objectives match the objectives in the Biology Teachers' Guide. The teacher should use both resources, side by side, when preparing activities.

Each activity description then lists the required materials. Instructions for the local manfacture of several items are given in the front of this book in the section called Manufacture of Apparatus. If an activity requires the use of materials from this section they will be marked with a star (*) in the materials list. The same is true for chemicals that are mentioned in the Sources of Chemicals section. For example, the materials list may look like this:
``Materials: beakers*, copper (II) sulphate*, plastic spoon''
If you see this, you can refer to the materials list to see a suggested method for making your own beakers. In the sources of chemicals list you will find a common place where copper (II) sulphate can be found. 

After listing the objectives and materials, the description lists any hazards associated with the activity and precautions teachers should take to minimize these hazards. Next are procedures, both for preparing the activity and for executing the experiment. While preparation steps are generally to be performed by the teacher, the activity steps are often to be performed by the students themselves.

The description next describes the expected results and what conclusions may be drawn from them. Then follow the instructions for cleaning up, including methods for disposing of any waste. The section closes with questions useful for guiding classroom discussion. Students should discuss these questions in groups and share their answers with the class.

Many of the activities also include a Notes section to provide the teacher with additional information about the activity. This information may be practical or theoretical.

\subsection*{Application of the Companion Guide}

This guide is written for teachers to acquire the knowledge and skills needed to lead students in hands-on science learning. While all of the experiments in this companion guide may be performed as demonstrations, the intention is for many of the activities to be performed by the students themselves, individually or in small groups, under the direction of the teacher.

To prepare for such lessons, the teacher should attempt these experiments first. Especially for teachers who are new to hands-on science experimentation, these experiments should, in themselves, provide a useful training. If there are multiple subject teachers at the school, they are encouraged to experiment together. Once the teacher has achieved comfort and proficiency with the given activity, the teacher should integrate the activity into relevant lesson plans.

The vision is not for students to be spectators of science, but players themselves.

Finally, the teacher is advised to not regard the activities in this book as the only possible activities nor even the only possible activities for these particular objectives. Every educator has ideas for effective teaching and new ideas are the substance of development. After trying an activity, the teacher is strongly encouraged to devise and attempt alternatives. When possible, teachers should collaborate with each other on such experiments, and share with each other the ideas they develop.

The vision is not for teachers to be passive implementors, but innovators themselves.
